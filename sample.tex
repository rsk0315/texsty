%! lualatex -shell-escape \\nonstopmode\\input sample.tex

\documentclass[lualatex, ja=standard]{ltjsarticle}

\usepackage{rsk0315}

\title{\relsize{3}Sample with Lua\LaTeX{}}

\begin{document}

\maketitle
\newpage

\section{Samples}\label{sec:sample}

\subsection{Mathematical formul\ae}

$$
f(a) = \frac{1}{2\pi i}\cdot \oint_{\gamma} \frac{f(z)}{z-a} \, \dd{z}
$$

{
  \newcommand{\tta}{\text{\texttt{0}}}
  \newcommand{\ttb}{\text{\texttt{1}}}
  $$
  L(\tta^*\ttb \tta^*)
  = \{\tta^m\ttb \tta^n \mid m, n\in \mathbb{N}\}
  \in \mathcal{REG}
  $$
}

\subsection{Pseudocodes}

\newcommand{\lpf}{\operatorname{lpf}}
\begin{alg}
  \caption{線形篩}
  \Function(\Comment*[f]{$O(n)$ time}){\Fn{Linear-sieve}{$n$}}{
    \Input{$n\in\mathbb{N}$}
    \Output{$(\lpf(i))_{i=0}^n \in \mathbb{N}^{n+1}$}
    {$p \gets ()$}\Comment*{a list of primes in ascending order}
    {$l \gets (1)_{i=0}^n$}\Comment*{$l_i = \lpf(i)$}
    \For{$i \gets (2, \dots, n)$}{
      \If{$l_i = 1$}{
        {$l_i \gets i$}\;
        {$p \xgets{\concat} (i)$}\;
      }
      \For{$j \gets p$}{
        \lIf{$j > \min\,\{l_i, \floor{n/i}\}$}{\Break}
        {$l_{i\cdot j} \gets j$}\;
      }
    }
    \Return{$l$}\;
  }
\end{alg}

\subsection{Source codes}

\begin{samplecode}{rust}{線形篩}
struct LinearSieve(Vec<usize>);

impl LinearSieve {
    pub fn new(n: usize) -> Self {
        let mut lpf = vec![1; n + 1];
        let mut primes = vec![];
        for i in 2..=n {
            if lpf[i] == 1 {
                lpf[i] = i;
                primes.push(i);
            }
            let lpf_i = lpf[i];
            for &j in primes.iter().take_while(|&&j| j <= lpf_i.min(n / i)) {
                lpf[i * j] = j;
            }
        }
        Self(lpf)
    }

    pub fn dp<T>(
        &self,
        zero: T,
        one: T,
        eq: impl Fn(&T, usize) -> T,
        gt: impl Fn(&T, usize) -> T,
    ) -> Vec<T> {
        let n = self.0.len() - 1;
        if n == 0 {
            return vec![zero];
        } else if n == 1 {
            return vec![zero, one];
        }

        let mut res = vec![zero, one];
        res.reserve(n + 1);
        for i in 2..=n {
            let lpf = self.0[i];
            let j = i / lpf;
            let tmp = if lpf == self.0[j] { eq(&res[j], lpf) } else { gt(&res[j], lpf) };
            res.push(tmp);
        }
        res
    }
}
\end{samplecode}

\samplecodeinput{rust}{sample.rs}

\subsection{Figures}

\newfontfamily\fakebolded{Junicode.ttf}[FakeBold=30]
\newcommand{\bordered}[1]{\tikz{\node[text=white]{\fakebolded#1};\node{#1}}}
\newcommand{\gpf}{\operatorname{gpf}}

\begin{figure}[h]
  \begin{center}
    \begin{tikzpicture}
    \graph[tree layout, sibling sep=0pt]{
      1 -> {
        2 -> {
          4 -> {
            8 -> 16,
            12,
            20
          },
          6 -> 18,
          10
        },
        2 -> 14,
        3 -> {
          9,
          15
        },
        5,
        7,
        e/...,
        19
      }
    };
    \begin{scope}[every node/.style={font=\scriptsize}]
      \path
      (1) -- node {\bordered{2}} (2)
      (1) -- node {\bordered{3}} (3)
      (2) -- node {\bordered{2}} (4)
      (1) -- node {\bordered{5}} (5)
      (2) -- node {\bordered{3}} (6)
      (1) -- node {\bordered{7}} (7)
      (4) -- node {\bordered{2}} (8)
      (3) -- node {\bordered{3}} (9)
      (2) -- node {\bordered{5}} (10)
      (4) -- node {\bordered{3}} (12)
      (2) -- node {\bordered{7}} (14)
      (3) -- node {\bordered{5}} (15)
      (8) -- node {\bordered{2}} (16)
      (6) -- node {\bordered{3}} (18)
      (1) -- node {\bordered{19}} (19)
      (4) -- node {\bordered{5}} (20)
      (1) -- node {\bordered{\dots}} (e)
      ;
    \end{scope}
    \end{tikzpicture}
    \caption{$n/{\gpf(n)} \to n$ ($1 < n\le 20$)に辺がある木。}
  \end{center}
\end{figure}

\newfontfamily\fakeboldedtt{TheSansMonoCd-W4SemiLight.otf}[FakeBold=30]
\renewcommand{\bordered}[1]{\tikz{\node[text=white]{\relsize{-1}\fakeboldedtt#1};\node{\relsize{-1}\texttt{#1}}}}

\begin{figure}[h]
  \begin{center}
    \begin{tikzpicture}
      \node[initial, state, accepting] (q_0) at (180:2) {$q_0$};
      \node[state] (q_1) at (252:2) {$q_1$};
      \node[state] (q_2) at (324:2) {$q_2$};
      \node[state] (q_3) at (36:2) {$q_3$};
      \node[state] (q_4) at (108:2) {$q_4$};
      \path[->]
      (q_0) edge [loop] node{\bordered{0}} (q_0)
      (q_0) edge node{\bordered{1}} (q_1)
      (q_1) edge node{\bordered{0}} (q_2)
      (q_1) edge node{\bordered{1}} (q_3)
      (q_2) edge node{\bordered{0}} (q_4)
      (q_2) edge node{\bordered{1}} (q_0)
      (q_3) edge [out=330, in=315, looseness=1.75] node{\bordered{0}} (q_1)
      (q_3) edge node{\bordered{1}} (q_2)
      (q_4) edge node{\bordered{0}} (q_3)
      (q_4) edge [loop] node{\bordered{1}} (q_4)
      ;
    \end{tikzpicture}
    \caption{2進法で解釈したときに5の倍数となる\code{0}/\code{1}列を受理するDFA。}
  \end{center}
\end{figure}

\subsection{Hyperlinks}

Section~\hyperref[sec:sample]{\ref{sec:sample}\textsuperscript{→ p.~\pageref{sec:sample}}}

\stdrust{\href{https://doc.rust-lang.org/std/vec/struct.Vec.html}{\textcolor{colorcoltext}{\code{std::vec::Vec}}}} (\url{https://doc.rust-lang.org/std/vec/struct.Vec.html})

\end{document}
